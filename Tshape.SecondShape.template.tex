%%%%%%%%%%%%%%%%%%%%%%%%%%%%%%%%%%%%%%%%%%%%%%%%%%%%%%%%%
% {{.Title}}
%%%%%%%%%%%%%%%%%%%%%%%%%%%%%%%%%%%%%%%%%%%%%%%%%%%%%%%%%

\begin{frame}[plain]{ {{.Title}} } \vspace{-0.1em}
	%%{\small
	已知:T形截面梁$b= {{.B}} ,h= {{.H}} ,b'_f= {{.Bff}} ,h'_f= {{.Hff}} mm^2$承受弯矩设计值$M= {{.M}} kN\cdot m$。
	混凝土强度等级{{.C}},受力纵筋采用{{.S}}级钢筋。取$a'_s={{.A}}$mm。

	求:所需受拉纵筋面积$A_s$
%%   	\begin{center}
%%	\scalebox{0.3}{\includegraphics{./figures/BasicJob1.png}}
%%	\end{center}
\end{frame}

%%%%%%%%%%%%%%%%%%%%%%%%%%%%%%%%%%%%%%%%%%%%%%%%%%%%%%%%%
% {{.Title}}--判断截面类型
%%%%%%%%%%%%%%%%%%%%%%%%%%%%%%%%%%%%%%%%%%%%%%%%%%%%%%%%%
\begin{frame}[plain]
	(1) 判断截面类型
	\uncover<2->{
\vspace{-0.8em} {\small 
\begin{align*}
	a_s &= {{.A}}mm \\
	h_0 &= h- a_s = {{.H}} - {{.A}} = {{.H_0}}mm \\
	M_j &= \alpha_1 f_c b'_f h'_f (h_0 - 0.5h'f) \\
	    &= 1.0\times {{.F_c}}\times {{.Bff}}\times {{.Hff}}\times( {{.H_0}} - 0.5\times {{.Hff}} ) N\cdot mm \\
	    &= {{printf "%4.1f" .M_xEqHff}} kN\cdot m \\
	M &= {{.M}} > M_j, \text{二类T形截面}
\end{align*} 
}}
\end{frame}


%%%%%%%%%%%%%%%%%%%%%%%%%%%%%%%%%%%%%%%%%%%%%%%%%%%%%%%%%
% {{.Title}}--计算受压区高度
%%%%%%%%%%%%%%%%%%%%%%%%%%%%%%%%%%%%%%%%%%%%%%%%%%%%%%%%%
\begin{frame}[plain]
	(2) 计算受压区高度
	\uncover<2->{
\vspace{-0.8em} {\small 
\begin{align*}
	M_1 &= \alpha_1 f_c (b'_f - b) h'_f (h_0 - 0.5h'f) \\
	    &= 1.0\times {{.F_c}}\times ( {{.Bff}} - {{.B}} )\times {{.Hff}} \times( {{.H_0}} - 0.5\times {{.Hff}} ) N\cdot mm\\
	    &= {{printf "%4.1f" .M_1}} kN\cdot m \\
	M_2 &= M - M_1 = {{.M}} - {{printf "%4.1f" .M_1}} = {{printf "%4.1f" .M_2}} kN\cdot m\\
	\alpha_s &= \frac{M_2}{\alpha_1 f_c b h^2_0} = \frac{ {{printf "%4.1f" .M_2}}\times10^6 }{ 1.0\times {{.F_c}}\times {{.B}}\times {{.H_0}}^2 }={{printf "%5.3f" .Alpha_s}} \\
	\xi &= 1 - \sqrt{1-2\alpha_s} = 1 - \sqrt{1-2\times {{printf "%5.3f" .Alpha_s}} } = {{printf "%5.3f" .Xi}} \\
	\xi_b &= {{.Xi_b}} > \xi, \text{受拉纵筋屈服} \\
	x &= \xi h_0 = {{printf "%5.3f" .Xi}}\times {{.H_0}} = {{printf "%4.1f" .X}} mm \\
\end{align*} 
}}
\end{frame}


%%%%%%%%%%%%%%%%%%%%%%%%%%%%%%%%%%%%%%%%%%%%%%%%%%%%%%%%%
% {{.Title}}--计算A_s
%%%%%%%%%%%%%%%%%%%%%%%%%%%%%%%%%%%%%%%%%%%%%%%%%%%%%%%%%
\begin{frame}[plain]
	(3) 计算$A_s$
\vspace{-0.8em} {\small 
\uncover<2->{
\begin{align*}
	A_s &= \frac{\alpha_1 f_c (b'_f -b) h'_f + \alpha_1 f_c b x}{f_y} \\
	    &=\frac{ 1.0\times {{.F_c}}\times ({{.Bff}} - {{.B}} )\times {{.Hff}} }{ {{.F_y}} } \\ 
	    &+ \frac{1.0\times {{.F_c}}\times {{.B}}\times {{printf "%4.1f" .X}} }{ {{.F_y}} } \\
	    &= {{printf "%5.1f" .A_s}} mm^2 
\end{align*}
}

\uncover<3-> {
	(4) 最小配筋率验算 \\
	最小配筋率计算,略。	
}
}
\end{frame}
