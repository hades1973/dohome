%%%%%%%%%%%%%%%%%%%%%%%%%%%%%%%%%%%%%%%%%%%%%%%%%%%%%%%%%
% 题1
%%%%%%%%%%%%%%%%%%%%%%%%%%%%%%%%%%%%%%%%%%%%%%%%%%%%%%%%%

\begin{frame}[plain]{ 题1 } \vspace{-0.1em}
	%%{\small
	已知:T形截面梁$b= 250 ,h= 700 ,b'_f= 600 ,h'_f= 100 mm^2$承受弯矩设计值$M= 550 kN\cdot m$。
	混凝土强度等级C30,受力纵筋采用HRB400级钢筋。取$a'_s=60$mm。

	求:所需受拉纵筋面积$A_s$
%%   	\begin{center}
%%	\scalebox{0.3}{\includegraphics{./figures/BasicJob1.png}}
%%	\end{center}
\end{frame}

%%%%%%%%%%%%%%%%%%%%%%%%%%%%%%%%%%%%%%%%%%%%%%%%%%%%%%%%%
% 题1--判断截面类型
%%%%%%%%%%%%%%%%%%%%%%%%%%%%%%%%%%%%%%%%%%%%%%%%%%%%%%%%%
\begin{frame}[plain]
	(1) 判断截面类型
	\uncover<2->{
\vspace{-0.8em} {\small 
\begin{align*}
	a_s &= 60mm \\
	h_0 &= h- a_s = 700 - 60 = 640mm \\
	M_j &= \alpha_1 f_c b'_f h'_f (h_0 - 0.5h'f) \\
	    &= 1.0\times 14.3\times 600\times 100\times( 640 - 0.5\times 100 ) N\cdot mm \\
	    &= 506.2 kN\cdot m \\
	M &= 550 > M_j, \text{二类T形截面}
\end{align*} 
}}
\end{frame}


%%%%%%%%%%%%%%%%%%%%%%%%%%%%%%%%%%%%%%%%%%%%%%%%%%%%%%%%%
% 题1--计算受压区高度
%%%%%%%%%%%%%%%%%%%%%%%%%%%%%%%%%%%%%%%%%%%%%%%%%%%%%%%%%
\begin{frame}[plain]
	(2) 计算受压区高度
	\uncover<2->{
\vspace{-0.8em} {\small 
\begin{align*}
	M_1 &= \alpha_1 f_c (b'_f - b) h'_f (h_0 - 0.5h'f) \\
	    &= 1.0\times 14.3\times ( 600 - 250 )\times 100 \times( 640 - 0.5\times 100 ) N\cdot mm\\
	    &= 295.3 kN\cdot m \\
	M_2 &= M - M_1 = 550 - 295.3 = 254.7 kN\cdot m\\
	\alpha_s &= \frac{M_2}{\alpha_1 f_c b h^2_0} = \frac{ 254.7\times10^6 }{ 1.0\times 14.3\times 250\times 640^2 }=0.174 \\
	\xi &= 1 - \sqrt{1-2\alpha_s} = 1 - \sqrt{1-2\times 0.174 } = 0.192 \\
	\xi_b &= 0.518 > \xi, \text{受拉纵筋屈服} \\
	x &= \xi h_0 = 0.192\times 640 = 123.2 mm \\
\end{align*} 
}}
\end{frame}


%%%%%%%%%%%%%%%%%%%%%%%%%%%%%%%%%%%%%%%%%%%%%%%%%%%%%%%%%
% 题1--计算A_s
%%%%%%%%%%%%%%%%%%%%%%%%%%%%%%%%%%%%%%%%%%%%%%%%%%%%%%%%%
\begin{frame}[plain]
	(3) 计算$A_s$
\vspace{-0.8em} {\small 
\uncover<2->{
\begin{align*}
	A_s &= \frac{\alpha_1 f_c (b'_f -b) h'_f + \alpha_1 f_c b x}{f_y} \\
	    &=\frac{ 1.0\times 14.3\times (600 - 250 )\times 100 }{ 360 } \\ 
	    &+ \frac{1.0\times 14.3\times 250\times 123.2 }{ 360 } \\
	    &= 2613.5 mm^2 
\end{align*}
}

\uncover<3-> {
	(4) 最小配筋率验算 \\
	最小配筋率计算,略。	
}
}
\end{frame}
%%%%%%%%%%%%%%%%%%%%%%%%%%%%%%%%%%%%%%%%%%%%%%%%%%%%%%%%%
% 题2
%%%%%%%%%%%%%%%%%%%%%%%%%%%%%%%%%%%%%%%%%%%%%%%%%%%%%%%%%

\begin{frame}[plain]{ 题2 } \vspace{-0.1em}
	%%{\small
	已知:T形截面梁$b= 250 ,h= 700 ,b'_f= 600 ,h'_f= 100 mm^2$承受弯矩设计值$M= 100 kN\cdot m$。
	混凝土强度等级C30,受力纵筋采用HRB400级钢筋。取$a'_s=60$mm。

	求:所需受拉纵筋面积$A_s$
%%   	\begin{center}
%%	\scalebox{0.3}{\includegraphics{./figures/BasicJob1.png}}
%%	\end{center}
\end{frame}

%%%%%%%%%%%%%%%%%%%%%%%%%%%%%%%%%%%%%%%%%%%%%%%%%%%%%%%%%
% 题2--判断截面类型
%%%%%%%%%%%%%%%%%%%%%%%%%%%%%%%%%%%%%%%%%%%%%%%%%%%%%%%%%
\begin{frame}[plain]
	(1) 判断截面类型
	\uncover<2->{
\vspace{-0.8em} {\small 
\begin{align*}
	a_s &= 60mm \\
	h_0 &= h- a_s = 700 - 60 = 640mm \\
	M_j &= \alpha_1 f_c b'_f h'_f (h_0 - 0.5h'_f) \\
	    &= 1.0\times 14.3\times 600\times 100\times( 640 - 0.5\times 100 ) N\cdot mm \\
	    &= 506.2 kN\cdot m \\
	M &= 100 < M_j, \text{第一类T形截面}
\end{align*} 
}}
\end{frame}


%%%%%%%%%%%%%%%%%%%%%%%%%%%%%%%%%%%%%%%%%%%%%%%%%%%%%%%%%
% 题2--计算受压区高度
%%%%%%%%%%%%%%%%%%%%%%%%%%%%%%%%%%%%%%%%%%%%%%%%%%%%%%%%%
\begin{frame}[plain]
	(2) 计算受压区高度
	\uncover<2->{
\vspace{-0.8em} {\small 
\begin{align*}
	\alpha_s &= \frac{M}{\alpha_1 f_c b'_f h^2_0} = \frac{ 100.0\times10^6 }{ 1.0\times 14.3\times 600\times 640^2 }=0.028 \\
	\alpha_{s,max} &= \xi_b(1 - 0.5\xi_b)= 0.518\times (1 - 0.5\times 0.518 ) = 0.384 \\
	\alpha_s &< \alpha_{s,max}, \text{受拉纵筋屈服} \\
	\xi &= 1 - \sqrt{1-2\alpha_s} = 1 - \sqrt{1-2\times 0.028 } = 0.029 \\
	x &= \xi h_0 = 0.029\times 640 = 18.5 mm \\
\end{align*} 
}}
\end{frame}


%%%%%%%%%%%%%%%%%%%%%%%%%%%%%%%%%%%%%%%%%%%%%%%%%%%%%%%%%
% 题2--计算A_s
%%%%%%%%%%%%%%%%%%%%%%%%%%%%%%%%%%%%%%%%%%%%%%%%%%%%%%%%%
\begin{frame}[plain]
	(3) 计算$A_s$
\vspace{-0.8em} {\small 
\uncover<2->{
\begin{align*}
	A_s &= \frac{\alpha_1 f_c b'_f x}{f_y} \\
	    &= \frac{1.0\times 14.3\times 600\times 18.5 }{ 360 } \\
	    &= 440.4 mm^2 
\end{align*}
}

\uncover<3-> {
	(4) 最小配筋率验算 \\
	最小配筋率计算,略。	
}
}
\end{frame}
